% Chapter Template

\chapter*{Introduction} % Main chapter title

\label{ch:Introduction} % Change X to a consecutive number; for referencing this chapter elsewhere, use \ref{ChapterX}

\lhead{\emph{Introduction}} % Change X to a consecutive number; this is for the header on each page - perhaps a shortened title

%----------------------------------------------------------------------------------------
%	INTRO TEXT
%----------------------------------------------------------------------------------------


The \textbf{user interface}, or human-computer interface (HCI) is the component of a computer system that provides a space of interaction between the human user and the resources offered by the machine; such a space defines a bridge language which human intentions can be translated into, to be converted into computational procedures for the machine; vice versa, the result of the computation is then presented to the user in the same language, which he or she is assumed to understand.

% cite Tanenbaum
In the early days of computing, the so-called \textbf{batch interfaces} were non-interactive: the user/programmer was supposed to feed the machine with a software, punched on cards using the \textit{machine's assembly language} directly, and retrieve the result of the computation printed on paper. The third and fourth generations of computing brought \textbf{text-based} interfaces for operating systems (UNIX, DOS, CP/M), that were later took over by \textbf{Graphical User Interfaces} (GUI), moving the human-machine interaction on a new, visual language made of windows, icons and buttons. Recently, touch screen and camera devices allowed for the implementation of even more natural means of interaction based on \textbf{gestures} and physical actions.

From this brief spot on computing history, and in a way from common sense, we can draw the rather trivial, yet crucial, conclusion that the trend in user interfaces development is to \textbf{close the gap} between humans and machines by moving the needle of the interface languages from a machine-centered space towards the human language itself. To this respect, the studies on Natural Language Processing (NLP) assume a dramatically central role, as dramatically central is natural language in the interaction of humans with each other.

%----------------------------------------------------------------------------------------
%	DIALOGUE SYSTEMS
%----------------------------------------------------------------------------------------

\section*{Dialogue Systems}

A spoken dialogue system, or conversational agent (CA) allow humans and machines to interact through an intermediate language which is as close as possible to the \textbf{human language}, and through conversational episodes that implement as close as possible the human dialogue modalities.

% Cite SHRDLU: http://dspace.mit.edu/handle/1721.1/7095?show=full
Research in dialogue systems has been carried on since the \textbf{early days} of Artificial Intelligence. A milestone in the early work on this field is ELIZA \cite{Weizenbaum:1966:ECP:365153.365168}, which provides the user with a basic human-like interaction based on pattern matching; another example is the SHRDLU system, which interfaces the user with a simple spatial domain being able to ambiguous or implicit references to the entities in it.

According to \cite{Jokinen2009}, modern dialogue systems can be divided in \textbf{two main types}: task-oriented and nontask-oriented. Intuitively, systems in the first category are meant to deal with a specific task such as making a hotel booking, or booking a plane ticket; an example in this category is the MIT Mercury system, a vocal interface to a flight database \cite{Seneff:2000:DMM:1605285.1605288}. On the other hand, nontask-oriented systems are meant to engage in conversations without a specific purpose to fulfill, but the one of delivering a realistic simulation; ELIZA itself is an example of nontask-oriented dialogue system.

\ldots

%----------------------------------------------------------------------------------------
%	LEARNING TO SPEAK
%----------------------------------------------------------------------------------------

\section*{Learning to speak}

\ldots

%----------------------------------------------------------------------------------------
%	THIS THESIS PROJECT
%----------------------------------------------------------------------------------------

\section*{This thesis project}

The aim of this thesis project is to design and implement language learning capabilities for an existing dialogue system, focusing on the \textbf{surface form} of the language.

This document is structured as follows. Chapter \ldots